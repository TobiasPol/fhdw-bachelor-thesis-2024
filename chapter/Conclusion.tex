\section{Conclusion}
% \addcontentsline{toc}{section}{Conclusion}
\fancyhead[R]{Conclusion}

\subsection{Summary of Findings}
\label{sec:Summary of Findings}

In this thesis, a novel Deep Learning model was developed to predict the enzymatic function based on the protein sequence. The research addressed the need for more accurate and efficient methods to predict enzyme classes responsible for pesticide degradation. The key findings of the research are summarized as follows:

\begin{enumerate}
    \item Novel Deep Learning Model: The developed Deep Learning model leverages enzyme binding site predictions to enhance the accuracy of enzyme class predictions. This approach focuses specifically on the ligand-binding sites, offering a more detailed and precise prediction by targeting critical interaction regions. With the use of the p2rank tool, the model can identify specific residues involved in catalytic processes, improving the accuracy of enzyme class predictions. Focussing on ligand-binding pockets has shown to be more effective than traditional methods that utilize the entire protein sequence.
    \item Data Preprocessing and Feature Engineering: Comprehensive data preprocessing steps, including cleaning, tokenization, and padding of sequences, ensured the dataset's quality and consistency. Feature engineering incorporated biochemical properties such as molecular weight, isoelectric point, hydrophobicity, and sequence length, which significantly contributed to the model's predictive power. The inclusion of these features enhance the model's ability to learn from the data and make accurate predictions.
    \item Handling Data Imbalance: The study addressed the dataset's imbalance using the RandomUnderSampler algorithm, which improved the model's ability to learn from underrepresented classes, although further optimization is necessary for perfect balance. There is a need for more data in UniProt to improve the model's performance, particularly for the 4th EC level, which has a large number of classes and is challenging to predict accurately. As shown in chapter \ref{sec:Data Preprocessing}
    \item Model Performance: The model demonstrated high predictive accuracy, particularly at the first four levels of the Enzyme Commission (EC) hierarchy. Existing models struggle to predict the 4th EC level due to the large number of classes and the dataset's imbalance. The developed model showed promising results at this level, indicating its potential to accurately predict enzyme classes in detail. The model's performance was further improved through hyperparameter tuning and data balancing techniques.
    \item Implications for Environmental Science: The findings highlight the model's potential to accelerate the development of environmentally friendly agricultural products by reducing the time and cost associated with experimental methods. Accurate enzyme class predictions facilitate better risk assessments and the development of sustainable bioremediation strategies.
\end{enumerate}

This thesis makes significant contributions to the fields of computational biology and environmental science by developing an innovative Deep Learning model. By leveraging modern Deep Learning techniques, this study enhances the accuracy and efficiency of enzyme function predictions, offering a substantial improvement over traditional methods. Moreover the model could save time and money in the development of new products, as well as reduce the risk of harmful pesticides in the environment.

The practical implications of this research are profound, particularly for sustainable agriculture. By enabling more precise predictions of enzyme-mediated pesticide degradation, the model supports the development of safer and more effective bioremediation strategies. This aligns with global efforts to reduce the environmental impact of pesticide use and promote sustainable agricultural practices. Moreover, the efficiency and accuracy of this model can lead to significant cost savings in the development and testing of new agricultural products, facilitating the faster introduction of environmentally friendly solutions to the market.

In addition to its environmental benefits, the research highlights the economic advantages of biotransformation. Utilizing enzymes to facilitate chemical reactions in pesticide degradation can significantly reduce production costs, making agricultural practices more sustainable and cost-effective. By predicting previously unknown enzyme classes involved in pesticide degradation, this model opens up new opportunities for developing innovative bioremediation solutions that are both environmentally friendly and economically viable. Traditional methods for determining enzyme classes are labor-intensive and time-consuming, making the model a valuable tool for accelerating the discovery of new enzymes and their functions.

\subsection{Final Remarks and Future Work}
\label{sec:Final Remarks and Future Work}

This thesis presents a novel Deep Learning model designed to predict enzyme classes based on their protein sequence. By leveraging detailed biochemical features and focusing on ligand-binding sites, the model demonstrates significant improvements on the first three levels and a slight increase on the 4th level in predictive accuracy compared to existing methods. The comprehensive data preprocessing, integration of sequence-based and property-based features, and the use of Deep Learning techniques such as LSTM layers have proven to be highly effective in capturing the complex nature of enzyme functions. The model's performance was further enhanced through hyperparameter tuning and data balancing techniques, demonstrating its potential to accurately predict enzyme classes. The implications of this research are far-reaching, with significant benefits for environmental science. Unknown enyzmes can be predicted accurately and used to develop new products. The model can also save time and money in the development of new products, as well as reduce the risk of harmful pesticides in the environment.

Despite the significant advancements presented in this thesis, there remains substantial potential for further improvement and refinement of the model. One potential enhancement is the integration of anchor sequences, as utilized in ECPred, to improve the decision-making power of the model. Anchor sequences are specific segments within proteins that are highly conserved and play crucial roles in their function. By focusing on these sequences, the model can gain a deeper understanding of the critical regions that determine enzyme activity. This approach can help in identifying key residues involved in substrate binding and catalysis, thereby improving the accuracy of enzyme classification. Anchor sequences provide essential structural and functional information that can enhance the model's ability to predict enzyme classes. They act as reliable markers for specific enzyme functions, allowing the model to make more informed predictions. Integrating anchor sequences into the feature set can help the model focus on the most relevant parts of the protein, improving its decision-making capabilities.

Another promising direction is the use of embeddings from ProtBERT, a pre-trained language model for protein sequences. ProtBERT embeddings capture rich contextual information from amino acid sequences, enabling the model to understand complex patterns and dependencies within the data. By incorporating these embeddings, the model can leverage the extensive knowledge encoded in ProtBERT, potentially enhancing its predictive performance ProtBERT embeddings provide a comprehensive representation of protein sequences, capturing both local and global sequence information. This can help the model generalize better across different enzyme classes and improve its ability to recognize subtle variations in sequence that are critical for enzyme function.

Further refinement of data augmentation techniques can address the issue of class imbalance more effectively. While RandomUnderSampling has been used in this study, exploring other methods such as SMOTE (Synthetic Minority Over-sampling Technique) or advanced generative models to create synthetic data for underrepresented classes can improve the model's performance on these classes. Also incorporating more diverse biochemical and environmental data can enhance the model's predictive accuracy. For example, integrating data on enzyme kinetics, thermodynamic stability, and environmental factors affecting enzyme activity can provide a more holistic view of enzyme function.

In conclusion, the Thesis represents a significant advancement in the field of enzyme classification. By integrating advanced Deep Learning techniques with detailed biochemical features, the model offers a powerful tool for computational biology. Future research should focus on incorporating anchor sequences, utilizing ProtBERT embeddings, refining data augmentation techniques, integrating additional data, and optimizing the model architecture to further enhance its capabilities and broaden its applicability.