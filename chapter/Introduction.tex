\section{Introduction}
% \addcontentsline{toc}{section}{Introduction}
\fancyhead[R]{Introduction}

\subsection{Motivation}
\label{sec:Motivation}

The prediction of pesticide degradation has gained significant attention due to its environmental and health impacts. Traditional methods for determining the degradation of pesticides by enzymes are laborintensive and time-consuming, necessitating the development of efficient computational methods. Pesticides, used globally for crop protection, often persist in the environment, posing risks to ecosystems and human health. Enzymes, as biological catalysts, play a crucial role in breaking down these pesticides into less harmful substances. This enzymatic degradation is essential for reducing the environmental toxicity of pesticides.

Recent advancements in DNA and RNA sequencing technologies have expanded the understanding of enzyme functions. Leveraging this wealth of data through computational methodologies, particularly Deep Learning, offers a promising approach to predicting enzymemediated pesticide degradation. Successful applications of Deep Learning in genomics and environmental science suggest its potential in enhancing pesticide degradation predictions, which is vital for developing sustainable agricultural products.

Especially in the context of discovering previously unknown enzymes and their functions, Deep Learning models can provide valuable insights into enzyme classification and behavior. By accurately predicting enzyme classes responsible for pesticide degradation, these models can facilitate the development of environmentally friendly and sustainable agricultural practices. The ability to predict enzyme functions with high accuracy and resolution is crucial for understanding the complex interactions between enzymes and pesticides, enabling the design of effective bioremediation strategies.

\subsection{Problem Statement}
\label{sec:Problem Statement}

Despite significant advancements in bioinformatics and computational biology, predicting enzyme classes involved in pesticide degradation remains challenging. Most of the traditional methods depend mainly on experimental data, which is time-consuming and high costly in most cases. Moreover, the functions of enzymes are very diverse and complex, as well as their interaction with different kinds of substrates, which further complicates the prediction process. Additionally, in some cases, the existing computational models have shown low prediction accuracies. This thesis puts forward an innovative Deep Learning model that employs predictions of enzyme binding sites to enhance the accuracy and efficiency of enzyme class prediction. The approach is focused on critical interaction regions, with the goal of outperforming state-of-the-art models and providing a strong tool for computational biology and environmental science. In particular, this research aims to:

\begin{enumerate}[label=(\alph*)]
    \item Develop a novel Deep Learning architecture that targets the ligand-binding sites of enzymes to predict the enyzmatic function.
    \item Enhance the accuracy and resolution of enzyme class predictions, particularly at the most specific level of the Enzyme Commission (EC) hierarchy.
    \item Address the challenges associated with data imbalance and class representation in enzyme classification datasets.
\end{enumerate}

\subsection{Structure of the Thesis}
\label{sec:Structure of the Thesis}

This thesis is divided into five chapters, each addressing different aspects of the research to overview it comprehensively. The first chapter of this thesis is an introduction to the entire work. It starts with outlining the motivation for study and states the environmental concerns related to pesticide use and the need for effective degradation prediction methods. This chapter includes the problem statement section, which identifies the problems associated with predicting enzyme-mediated pesticide degradation. The introduction defines the general objective of this study: developing a deep-learning model to predict pesticide degradation based on enzyme classes. Lastly, this chapter concludes with an overview of the thesis structure.

The literature review chapter covers existing research studies and foundational theories that are important to the study. It deals with enzymatic mechanisms under the degradation of pesticides, giving information on how the enzymes mediate the degradation process. This is succeeded by using Deep Learning techniques for application in environmental science and the improvements that it may offer to predictive accuracy. Finally, it suggests advanced methods that can be used in enzyme classification due to the limitations of the current models in many ways.

The methodology chapter offers a complete account of the research design and procedures followed in the study. It starts with data collection, stating the sources and pre-processing processes executed to prepare the dataset for analysis. The feature engineering section elaborates on how relevant features were engineered out of the data to enable accurate predictions. It is then followed by the model development process, the architecture of a Deep Learning model, and the final training process.

Chapter results talk about the research outcomes, starting with the evaluation of model performance by giving details of different model performance metrics and how effectively they predict pesticide degradation. It will also compare the developed model with those already in existence to demonstrate how improvements and benefits have been achieved. It interprets model predictions, including their practical implications, and applies them in real-world situations.

The discussion chapter will synthesize critical findings from the research and present a reflection on the significance and impact of these results. It discusses the strengths and limitations of the study by pointing out where models performed well and stating where there is room for improvement. The chapter ends with brief contributions to the field, pointing at the novelty and practical applications that the research relates to. In addition, it also gives future work recommendations in which the study points out research directions for building over its findings.