%!TEX root = ../Thesis.tex
%% Basierend auf TeXnicCenter-Vorlage von Mark Müller
%%                      Willi Nüßer
%%                      Waldemar Penner     
%%                      Ulrich Reus
%%                      Frank Plass
%%                      Oliver Tribeß 
%%                      Daniel Hintze     
%%%%%%%%%%%%%%%%%%%%%%%%%%%%%%%%%%%%%%%%%%%%%%%%%%%%%%%%%%%%%%%%%%%%%%%

% Wählen Sie die Optionen aus, indem Sie % vor der Option entfernen  
% Dokumentation des KOMA-Script-Packets: scrguide

%%%%%%%%%%%%%%%%%%%%%%%%%%%%%%%%%%%%%%%%%%%%%%%%%%%%%%%%%%%%%%%%%%%%%%%
%% Optionen zum Layout des Artikels                                  %%
%%%%%%%%%%%%%%%%%%%%%%%%%%%%%%%%%%%%%%%%%%%%%%%%%%%%%%%%%%%%%%%%%%%%%%%
\documentclass[%
paper=A4,         % alle weiteren Papierformat einstellbar
fontsize=12pt,    % Schriftgröße (12pt, 11pt (Standard))
BCOR=12mm,         % Bindekorrektur, bspw. 1 cm
DIV=14,            % breiter Satzspiegel
parskip=half*,    % Absatzformatierung s. scrguide 3.1
headsepline,      % Trennline zum Seitenkopf  
%footsepline,     % Trennline zum Seitenfuß
%normalheadings,  % Überschriften etwas kleiner (smallheadings)
listof=totoc,     % Tabellen & Abbildungsverzeichnis ins Inhaltsverzeichnis      
%bibtotoc,        % Literaturverzeichnis im Inhalt 
%draft            % Überlangen Zeilen in Ausgabe gekennzeichnet
footinclude=false,% Fußzeile in die Satzspiegelberechnung einbeziehen 
headinclude=true, % Kopfzeile in die Satzspiegelberechnung einbeziehen 
final             % draft beschleunigt die Kompilierung
]
{scrartcl}

%\setuptoc{toc}{totoc} % Inhaltsverzeichnis ins Inhaltsverzeichnis

% Neue Deutsche Rechtschreibung und Deutsche Standardtexte
\usepackage[ngerman]{babel} 

% Umlaute können verwendet werden
\usepackage[utf8]{inputenc}   

% Echte Umlaute
\usepackage[T1]{fontenc} 

% Latin Modern Font, Type1-Schriftart für nicht-englische Texte
\usepackage{lmodern} 

% 1/2-zeiliger Zeilenabstand
\usepackage[onehalfspacing]{setspace}

% Für die Defenition eigener Kopf- und Fußzeilen
\usepackage{fancyhdr} 

% Für die Verwendung von Grafiken
\usepackage[pdftex]{graphicx}

% Bessere Tabellen
\usepackage{tabularx}

% Für die Befehle \toprule, \midrule und \bottomrule, z.B. in Tabellen 
\usepackage{booktabs}

% Erlaubt die Benutzung von Farben
\usepackage{color}

% Verbessertes URL-Handling mit \url{http://...}
\usepackage{url}

% Listen ohne Abstände \begin{compactlist}...\end{compactlist}
\usepackage{paralist} 

% Ausgabe der aktuellen Uhrzeit für die Draft-Versionen
\usepackage{datetime}

% Deutsche Anführungszeichen
\usepackage[babel,german=quotes]{csquotes}

% Konfiguration der Abbildungs- und Tabellenbezeichnungen
\usepackage[format=hang, font={footnotesize, sf}, labelfont=bf, justification=raggedright,singlelinecheck=false]{caption}

% Verbessert die Lesbarkeit durch Mikrotypografie
\usepackage[activate={true,nocompatibility},final,tracking=true,kerning=true,spacing=true,factor=1100,stretch=10,shrink=10]{microtype}  

% Zitate und Quellenverzeichnis
\usepackage[
    bibstyle=authoryear,
    citestyle=authoryear-fhdw,  
    giveninits=false,         % false = Vornamen werden ausgeschrieben
    natbib=true,
    urldate=long,             % "besucht am" - Datum
    %url=false,
    date=long,                
    dashed=false, 
    maxcitenames=3,           % max. Anzahl Autorennamen in Zitaten
    maxbibnames=99,           % max. Anzahl Autorennamen im Quellenverzeichnis
    %backend=bibtex           % Ggf. für ältere Distributionen bibtex verwenden
    backend=biber
]{biblatex}
  
% Bibliograpthy
\bibliography{library/Bachelor Thesis}

% Keine Einrückung bei einem neuen Absatz 
\parindent 0pt 

% Ebenentiefe der Nummerierung
\setcounter{secnumdepth}{3}

% Gliederungstiefe im Inhaltsverzeichnis 
\setcounter{tocdepth}{3} 

% Tabellen- und Abbildungsverzeichnis mit Bezeichnung:
\usepackage[titles]{tocloft}

% Sourcecode-Listings
\usepackage{listings}

% Bestimmte Warnungen unterdrücken
% siehe http://tex.stackexchange.com/questions/51867/koma-warning-about-toc
\usepackage{scrhack} 

%% http://tex.stackexchange.com/questions/126839/how-to-add-a-colon-after-listing-label
\makeatletter
\begingroup\let\newcounter\@gobble\let\setcounter\@gobbletwo
  \globaldefs\@ne \let\c@loldepth\@ne
  \newlistof{listings}{lol}{\lstlistlistingname}
\endgroup
\let\l@lstlisting\l@listings
\makeatother

\renewcommand*\cftfigpresnum{Abbildung~}
\renewcommand*\cfttabpresnum{Tabelle~}
\renewcommand*\cftlistingspresnum{Listing~}
\renewcommand{\cftfigaftersnum}{:}
\renewcommand{\cfttabaftersnum}{:}
\renewcommand{\cftlistingsaftersnum}{:}
\settowidth{\cftfignumwidth}{\cftfigpresnum 99~\cftfigaftersnum}
\settowidth{\cfttabnumwidth}{\cfttabpresnum 99~\cftfigaftersnum}
\settowidth{\cftlistingsnumwidth}{\cftlistingspresnum 99~\cftfigaftersnum}
\setlength{\cfttabindent}{1.5em}
\setlength{\cftfigindent}{1.5em}
\setlength{\cftlistingsindent}{1.5em}

\renewcommand\lstlistlistingname{Listingverzeichnis}

% Style für Kopf- und Fußzeilenfelder
\pagestyle{fancy}
\fancyhf{}
\fancyhead[R]{\leftmark}
\fancyfoot[R]{\thepage} 
\renewcommand{\sectionmark}[1]{\markboth{#1}{#1}} 
\fancypagestyle{plain}{}

% Macro für Quellenangaben unter Abbildungen und Tabellen
\newcommand{\source}[1]{{\vspace{-1mm}\\\footnotesize\textsf{\textbf{Quelle:}} \textsf{#1}\par}}

% Anpassungen der Formatierung an Eclipse-Aussehen 
% http://jevopi.blogspot.de/2010/03/nicely-formatted-listings-in-latex-with.html
%\definecolor{sh_comment}{rgb}{0.12, 0.38, 0.18 } %adjusted, in Eclipse: {0.25, 0.42, 0.30 } = #3F6A4D
%\definecolor{sh_keyword}{rgb}{0.37, 0.08, 0.25}  % #5F1441
%\definecolor{sh_string}{rgb}{0.06, 0.10, 0.98} % #101AF9
% Für Druckausgabe sollte alles schwarz sein
\definecolor{sh_comment}{rgb}{0.0, 0.0, 0.0 }
\definecolor{sh_keyword}{rgb}{0.0, 0.0, 0.0 }
\definecolor{sh_string}{rgb}{0.0, 0.0, 0.0 }

\lstset{ %
  language=Python,
  basicstyle=\small\ttfamily,
  fontadjust, 
  xrightmargin=1mm,
  xleftmargin=5mm,
  tabsize=2,
  columns=flexible,
  showstringspaces=false,
  rulesepcolor=\color{black},
  showspaces=false,showtabs=false,tabsize=2,
  stringstyle=\color{sh_string},
  keywordstyle=\color{sh_keyword}\bfseries,
  commentstyle=\color{sh_comment}\itshape,
  captionpos=t,
  lineskip=-0.3em
}

%\makeatletter
%\def\l@lstlisting#1#2{\@dottedtocline{1}{0em}{1.5em}{\lstlistingname\space{#1}}{#2}}
%\makeatother

% Anhangsverzeichnis
\usepackage[nohints]{minitoc} %Anhangsverzeichnis

\makeatletter
\newcounter{fktnr}\setcounter{fktnr}{0}
\newcounter{subfktnr}[fktnr]\setcounter{subfktnr}{0}

\renewcommand\thesubfktnr{\arabic{fktnr}.\arabic{subfktnr}}
\newcounter{anhangcounter}
\newcommand{\blatt}{\stepcounter{anhangcounter}}

\newcommand{\anhang}[1]{\setcounter{anhangcounter}{0}\refstepcounter{fktnr}
\addcontentsline{fk}{subsection}{Anhang~\thefktnr: \hspace*{1em}#1}
\subsection*{{Appendix~\thefktnr \hspace*{1em} #1 \hspace*{-1em}}}
}

\newcommand{\subanhang}[1]{\setcounter{anhangcounter}{0}\refstepcounter{subfktnr}
\addcontentsline{fk}{subsubsection}{Anhang~\thesubfktnr: \hspace*{1em}#1}
\subsubsection*{{Appendix~\thesubfktnr \hspace*{1em} #1 \hspace*{-1em}}}
}

\newcommand{\anhangsverzeichnis}{\mtcaddsection{\subsection*{Appendix \@mkboth{FKT}{FKT}}}\@starttoc{fk}\newpage}

% Links im PDF
\usepackage[pdfpagemode={UseOutlines}, plainpages=false,breaklinks=true,pdfpagelabels]{hyperref}

 % Abkürzungsverzeichnis
\usepackage[automake,
			acronym,         % create list of acronyms
            nonumberlist,
            toc, 
            section,
            nopostdot,  % avoid dot after acronym
            hyperfirst=false,% don't hyperlink first use
            %sanitize=none    % switch off sanitization as description % Deprecated
            ]{glossaries}
            \newglossarystyle{mylist}{%
\setglossarystyle{long}% base this style on the list style
\renewcommand*{\glossaryentryfield}[5]{%
    \glsentryitem{##1}\textbf{##2} & ##3 \\}%
}

% Verbessert das Referenzieren von Kapiteln, Abbildungen etc.
\usepackage[english,capitalise]{cleveref}

\makeglossaries\makeglossaries
%Acronyms
\newacronym{AES}{AES}{Advanced Encryption Standard}
\newacronym{AI}{AI}{Artificial Intelligence}
\newacronym{AOA}{AOA}{Angle of Arrival}
\newacronym{API}{API}{Application Programming Interface}
\newacronym{ATM}{ATM}{Automated Teller Machine}

%Glossary
\newglossaryentry{Glossar}
{
	name=Glossar,
	description={Ein Glossar ist eine alphabetisch geordnete Liste von Begriffen aus einem bestimmten Wissensgebiet mit den dazugehörigen Definitionen.}
}
