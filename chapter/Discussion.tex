\section{Discussion}
% \addcontentsline{toc}{section}{Discussion}
\fancyhead[R]{Discussion}

\subsection{Implications of Findings}
\label{sec:Implications of Findings}

The findings of this study have substantial implications for both the field of computational biology and the practical application of deep learning models in environmental science. By developing a deep learning model that accurately predicts enzyme classes responsible for pesticide degradation, this research contributes to several critical areas.

Traditional methods of determining pesticide degradation and enzyme classification are often labor-intensive, time-consuming, and expensive. The computational approach presented in this study offers a more efficient alternative. By leveraging deep learning models, the research significantly reduces the time and cost associated with experimental methods. This efficiency can accelerate the development and testing of new agricultural products, ensuring that safer and more effective solutions reach the market faster.

The model developed in this study demonstrates significant improvements in predictive accuracy, particularly for the 4th level of the Enzyme Commission EC classification hierarchy. This enhanced accuracy is crucial for advancing the understanding of enzyme functions and their specific roles in biodegradation processes. Accurately predicting enzyme classes allows for more precise identification of enzymatic pathways involved in pesticide degradation, which is fundamental for developing effective bioremediation strategies. The Deep Learning model can be employed to help predicting unknown enzyme classifications in order to clean up contaminated environments more efficiently and reduce the ecological footprint of agricultural practices. The findings support the creation of more sustainable and environmentally friendly agricultural products, aligning with global efforts to mitigate pollution and protect natural ecosystems. At Bayer Crop Science, the model can be integrated into the product development process to enhance the safety and sustainability of new agricultural products, reducing the environmental impact of pesticide use. Especially in the modern context of increasing environmental awareness and regulatory scrutiny, the model provides a valuable tool for the development of new enzymes and biodegradation pathways.

The findings open several avenues for future research. One potential direction is the refinement of the model to improve performance at the fourth EC level, which remains challenging due to the high specificity and diversity of enzyme functions. Additionally, integrating this model with real-world environmental data can validate its practical applicability and uncover further insights into enzymatic degradation pathways. Collaborative efforts with experimental biologists can enhance the model's accuracy and expand its scope to include a wider range of pollutants and environmental conditions.

Moreover, this novel approach can be further improved to achieve even better results. Current methods often utilize the entire protein sequence and do not focus specifically on the ligand-binding pocket. By concentrating more on these specific pockets, it is possible to enhance the precision of enzyme classification and the prediction of degradation pathways. Future advancements should therefore aim to refine this focus on ligand-binding sites, leveraging detailed structural information to improve predictive accuracy.


\subsection{Strenths and Limitations}
\label{sec:Strenths and Limitations}

\textbf{Strengths:}

\begin{enumerate}
    \item Innovative Approach: The primary strength of this thesis lies in its innovative approach to predicting pesticide degradation by focusing on enzyme classification through deep learning. By integrating advanced computational techniques, this research addresses a gap in the current methodologies used for enzyme function prediction.
    \item Comprehensive Methodology: The detailed and methodical approach taken in data collection, preprocessing, feature engineering, and model development ensures the robustness of the study. Each step is meticulously documented, demonstrating a thorough understanding of the processes involved in developing a predictive model.
    \item Utilization of Advanced Tools: The use of state-of-the-art tools such as P2Rank for ligand-binding site prediction and RNNs for sequence analysis highlights the technical sophistication of the study. These tools provide a solid foundation for accurate predictions and demonstrate the potential for further applications in bioinformatics.
    \item Significant Performance Improvement: The developed model shows significant improvement in predictive accuracy, particularly at the higher levels of the Enzyme Commission hierarchy. This improvement underscores the effectiveness of combining sequence-based features with additional biochemical features.
    \item Environmental and Economic Impact: By enabling more accurate predictions of pesticide degradation pathways, the study contributes to environmental sustainability and cost efficiency. The ability to develop targeted bioremediation techniques and accelerate the development of environmentally friendly agricultural products has far-reaching benefits.
\end{enumerate}

\textbf{Limitations:}

\begin{enumerate}
    \item Performance at Specific EC Levels: While the model performs well at higher EC levels, there is a notable decrease in performance at the most specific level (level 4). This limitation suggests that the model struggles with the high specificity and diversity of enzyme functions at this level, necessitating further refinement and optimization. One of the reasons is that the gls{gls{gls{UniProt}}} database is still unbalanced and needs to be further improved. In addition to that, it is possible that some enzymes are wrongly classified in the database, which can lead to wrong predictions.
    \item Imbalanced Dataset: The initial dataset used in the study is highly imbalanced, with certain enzyme classes being significantly underrepresented. Although techniques such as SMOTE were employed to address this issue, the imbalance may still affect the model's ability to generalize across all enzyme classes.
    \item Focus on Ligand-Binding Pockets: Although the study emphasizes the importance of ligand-binding pockets, current methods still utilize the entire protein sequence, which may dilute the specificity of predictions. Future research should aim to enhance the focus on these pockets to improve predictive accuracy.
    \item Generalizability to Real-World Data: The model's performance is primarily evaluated using data from UniProt and PDB, which are well-curated databases. The generalizability of the model to real-world environmental data remains to be validated, as real-world scenarios often involve more complex and noisy data. Although the model needs to be validated in vivo or in vitro, the results are promising and provide a strong foundation for further research.
\end{enumerate}