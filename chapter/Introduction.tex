\section{Introduction}
% \addcontentsline{toc}{section}{Introduction}
\fancyhead[R]{Introduction}

\subsection{Motivation}
\label{sec:Motivation}

In recent years, the prediction of pesticide degradation has gained significant attention due to its environmental and health impacts. Traditional methods for determining the degradation of enzymes are labor-intensive and time-consuming. Consequently, there is a growing need for computational methods that can efficiently and accurately predict the degradation behavior of pesticides.

The pesticide degradation can be facilitated by enzymes, which are biological catalysts that accelerate chemical reactions. Enzymes play a crucial role in breaking down pesticides into harmless byproducts, reducing their toxicity and environmental impact. Understanding the enzymatic mechanisms involved in pesticide degradation is essential for developing sustainable and environmentally friendly agricultural products.

Recent advancements in DNA and RNA sequencing technologies have led to an explosion of information about new organisms and their enzymes. These developments are crucial for the industry and, in particular, for Bayer, as they provide a wealth of data that can be leveraged to enhance the prediction and understanding of enzymatic functions involved in pesticide degradation. This influx of sequence data from diverse ecosystems, including soil, offers new opportunities to identify enzymes that play significant roles in environmental processes.

The applications of advanced computational methodologies, such as Deep Learning for predicting these enzymatic functions, have the potential to significantly enhance the development of environmentally friendly and safe agricultural products at Bayer Crop Science. This would also reduce the time and cost of testing existing and new products, as well as the risk of developing products that are harmful to the environment. Despite significant advancements in bioinformatics and computational biology, predicting enzyme classes remains challenging and fraught with uncertainties. Traditional methods rely heavily on experimental data, which can be resource-intensive and time-consuming. Moreover, the vast diversity of enzyme functions and their complex interactions with various substrates add layers of difficulty to accurate predictions. Thus, there is a pressing need for computational tools that leverage modern machine learning techniques to enhance the prediction accuracy of enzyme-related models.

This thesis introduces a novel Deep Learning model that leverages enzyme binding site predictions to enhance the accuracy and efficiency of enzyme class prediction. Unlike existing models, the approach specifically targets enzyme binding sites, offering a more detailed and accurate prediction by focusing on the critical interaction regions.

By addressing this research question, the study seeks to contribute to the fields of computational biology and environmental science, providing a tool that can accurately predict enzymatic functions and their behavior in pesticide degradations. In addition, this research aims to outperform existing models in predicting enzyme classes responsible for pesticide degradation, thereby enhancing the accuracy of enzyme classification predictions. The findings of this study could have significant implications for the development of environmentally friendly and sustainable agricultural products, as well as the reduction of harmful pesticides in the environment.

\subsection{Structure of the Thesis}
\label{sec:Structure of the Thesis}
This thesis is structured into five chapters, each addressing different aspects of the research and providing a comprehensive overview of the study.
The first chapter sets the stage for the entire thesis. It begins by outlining the motivation behind the research, highlighting the environmental concerns related to pesticide use and the need for effective degradation prediction methods. The problem statement section identifies the challenges associated with predicting enzyme-mediated pesticide degradation. The introduction section defines the main objective of the study, which is to develop a Deep Learning model to predict pesticide degradation based on enzyme classes. Finally, this chapter provides an overview of the structure of the thesis.

The literature review chapter delves into existing research and foundational theories relevant to the study. It covers enzymatic mechanisms involved in pesticide breakdown, offering insights into how enzymes facilitate the degradation process. Additionally, it explores the application of Deep Learning techniques in environmental science, emphasizing their potential to enhance predictive accuracy. The chapter concludes with a discussion of the limitations of current models and the need for more advanced approaches to enzyme classification.

The methodology chapter provides a detailed description of the research design and procedures followed in this study. It begins with the Data Collection, specifying the sources and preprocessing steps to prepare the dataset for analysis. The feature engineering section discusses how relevant features were extracted from the data to calculate accurate predictions. The chapter then explains the model development process, including the architecture of the Deep Learning model and the final training process.

The results chapter presents the outcomes of the research. It begins with an evaluation of the model's performance, highlighting key metrics and the effectiveness of the model in predicting pesticide degradation. A comparative analysis with existing models is included to demonstrate the improvements and advantages of the developed model. The chapter also interprets the model predictions, offering insights into the practical implications of the findings and how they can be applied in real-world scenarios.

The discussion chapter summarizes the key findings of the research, reflecting on the significance and impact of the results. It discusses the strengths and limitations of the study, acknowledging areas where the model performed well and identifying potential areas for improvement. The chapter concludes with an overview of the contributions to the field, highlighting the novelty and practical applications of the research. Additionally, it provides recommendations for future work, suggesting directions for further research to build on the findings of this study.