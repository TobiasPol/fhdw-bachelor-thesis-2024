\section{Literature Review}
% \addcontentsline{toc}{section}{Literature Review}
\fancyhead[R]{Literature Review}

\subsection{Enzymatic Mechanisms Involved in Pesticide Breakdown}
\label{sec:Enzymatic Mechanisms Involved in Pesticide Breakdown}

The breakdown of pesticides in the environment is a complex process involving various mechanisms, primarily driven by microbial enzymes. These enzymes catalyze reactions that convert toxic pesticide compounds into less harmful substances, facilitating their removal from the environment. This section explores the key enzymatic mechanisms involved in pesticide degradation, focusing on hydrolytic, oxidative, and reductive enzymes.

Microbial enzymes play a pivotal role in the biodegradation of soil contaminants, including pesticides. They can be categorized based on the reactions they catalyze:

Hydrolytic Enzymes: Hydrolytic enzymes, such as esterases and amidases, catalyze the cleavage of ester and amide bonds in pesticide molecules. This hydrolysis results in the formation of smaller, more water-soluble compounds that are easier to further degrade and eliminate. For example, microbial esterases can hydrolyze organophosphate insecticides, significantly accelerating their breakdown.

Oxidative Enzymes: Oxidative enzymes, such as cytochrome P450 monooxygenases, introduce oxygen atoms into the pesticide molecules, increasing their solubility and reactivity. This oxidation process often converts the pesticides into less harmful substances or intermediates that can be further degraded by other enzymes. The cytochrome P450 enzymes are particularly versatile, capable of metabolizing a wide range of xenobiotics, including pesticides.

Reductive Enzymes: Reductive enzymes, including reductases, catalyze the reduction of pesticides, often by adding electrons and hydrogen atoms to the molecules. This reduction can break down complex structures and facilitate the conversion of pesticides into simpler, less toxic forms. Reductive dehalogenases, for instance, play a significant role in the degradation of halogenated organic compounds.

Incorporating microbial enzymes into bioremediation strategies can significantly enhance the degradation of pesticides in contaminated soils. This approach leverages the natural capabilities of microbes to detoxify pollutants through enzymatic reactions. According to a review on the function of microbial enzymes in breaking down soil contaminated with pesticides, these enzymes are highly effective in transforming and mineralizing pesticides, thus reducing their environmental impact. \footcite{singhMicrobialDegradationOrganophosphorus2006}

Another study highlights the advancements and applications of microbial enzymes in biodegradation processes. This review emphasizes the critical role of enzymes in the degradation pathways of various pesticides and discusses the potential for engineered enzymes to improve bioremediation efficiency. \footcite{chiaFunctionMicrobialEnzymes2024}

\subsection{Deep Learning Techniques in Environmental Science}
\label{sec:Deep Learning Techniques in Environmental Science}

Deep learning has emerged as a powerful tool in environmental science, offering advanced methods for predicting and understanding complex biochemical processes. In the context of pesticide degradation, deep learning models can analyze vast amounts of biochemical data to predict enzyme interactions and degradation pathways. There are several deep learning architectures that have been successfully applied to enzyme classification and prediction tasks, providing valuable insights into the mechanisms of pesticide breakdown.

For instance, the DEEPre model uses deep learning to predict enzyme commission (EC) numbers from raw sequence data. This model has shown significant improvements in prediction accuracy over traditional methods by utilizing convolutional and sequential feature extraction techniques. Such models can be crucial for predicting the biodegradation pathways of pesticides, enabling more accurate and efficient environmental risk assessments. \footcite{liDEEPreSequencebasedEnzyme2017}

Another example is the DeEPn model, which uses a deep neural network to classify enzymes into their functional classes, including all seven EC classes. This model has demonstrated high precision and accuracy, making it a valuable tool for environmental scientists looking to understand and predict enzyme-mediated degradation processes. By accurately classifying enzymes, DeEPn facilitates the identification of potential candidates for bioremediation and other environmental applications. \footcite{DeEPnDeepNeural}

Despite the advances made by these models, there is still a need for new approaches to further improve the accuracy and applicability of pesticide degradation predictions. Traditional models often rely on pre-defined features and limited datasets, which can restrict their performance and generalizability. By contrast, my proposed approach leverages the deep learning tool p2rank to analyze the interactive parts of enzymes, focusing on the ligand-binding sites and the specific amino acids involved. \footcite{krivakP2RankMachineLearning2018} This method can potentially provide a more detailed and accurate prediction of enzyme classes responsible for pesticide degradation, enhancing our understanding of the biodegradation pathways and mechanisms involved.