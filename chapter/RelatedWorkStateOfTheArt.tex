\section{Related Work and State of the Art}
% \addcontentsline{toc}{section}{Related Work and State of the Art}
\fancyhead[R]{Related Work and State of the Art}

\subsection{Role of Enzymes in Environmental Pesticide Degradation}
\label{sec:Role of Enzymes in Environmental Pesticide Degradation}

Enzymes are biological catalysts that accelerate chemical reactions in living organisms. They are highly specific, meaning each enzyme typically catalyzes only one type of reaction or interacts with a specific substrate. This specificity arises from the enzyme’s unique three-dimensional structure, which includes an active site where the substrate binds and the reaction occurs. Enzymes function by lowering the activation energy required for a reaction, thus increasing the reaction rate without being consumed in the process. Their activity is influenced by various factors, including temperature, pH, and the presence of inhibitors or activators. \autocite{robinsonEnzymesPrinciplesBiotechnological2015}

The prediction of pesticide degradation and the identification of enzymatic functions involved is a critical area of research with significant implications for environmental sustainability and agricultural practices. Pesticides, used globally to protect crops, often persist in the environment, leading to potential ecological and health risks. Enzymes, as biological catalysts, play a pivotal role in degrading these toxic compounds into less harmful substances. The enzymatic degradation process involves various mechanisms, predominantly microbial enzymatic activities, which catalyze reactions that transform pesticides.

For instance, reductive enzymes catalyze the reduction of pesticides, often by donating electrons and hydrogen atoms to the molecules, breaking down complex structures into simpler forms. Reductive dehalogenases, found in microorganisms like Dehalococcoides, are particularly effective in breaking down halogenated organic compounds. The efficiency of microbial enzymes in degrading soil-contaminated pesticides has been well-documented. Singh and Walker (2012) \autocite{singhMicrobialDegradationOrganophosphorus2006} highlighted the effectiveness of microbial degradation of organophosphorus compounds, while Chia et al. (2013) \autocite{chiaFunctionMicrobialEnzymes2024} discussed advancements in microbial enzymes for enhancing biodegradation processes. Understanding these enzymatic mechanisms is crucial for predicting enzyme classes responsible for pesticide degradation and developing accurate predictive models.

Understanding these enzymatic mechanisms is crucial for predicting the enzyme classes responsible for pesticide degradation. By analyzing enzyme-pesticide interactions, it is possible to identify specific enzyme classes involved in the degradation processes. This knowledge can inform the development of more accurate predictive models for pesticide degradation, facilitating better risk assessments and environmental management strategies. Advanced computational methods, such as Deep Learning, can further enhance these predictive models by accurately identifying and classifying enzymes based on their interaction with pesticides, leading to more efficient and targeted development of new products.

\subsection{Deep Learning Techniques in Environmental Science}
\label{sec:Deep Learning Techniques in Environmental Science}

Deep Learning has become an essential tool in environmental science, enabling advanced prediction and understanding of complex biochemical processes. Various Deep Learning architectures, such as the protein-transformer ESM model, have significantly impacted the prediction of biological properties from sequence data. These models can analyze vast quantities of biochemical data to predict enzyme interactions and functions, providing valuable insights into pesticide degradation mechanisms. \autocite{rivesBiologicalStructureFunction2021}

In the context of pesticide degradation and enzyme classification, such models can analyze large quantities of available biochemical data to make predictions about enzyme interactions and functions. Several Deep Learning architectures have been applied in enzyme classification and prediction tasks, from which valuable insights into the mechanism of pesticide degradation can be obtained.

For instance, the DEEPre model applies Deep Learning to predict EC numbers based on raw sequence data. Such models apply convolutional and sequential feature extraction techniques, leading to significant improvements in prediction accuracy over methods in current use. In this respect, such models may play a key role in predicting the pesticide biodegradation pathways and help to make environmental risk assessment more precise and fast. \autocite{liDEEPreSequencebasedEnzyme2017}

Despite the advances made by these models, there is still a need for new approaches to further improve the accuracy of sequence based predictions. Traditional models often rely on pre-defined features and limited datasets, which can restrict their performance and generalizability. In addition to this, the existing methods only focus on the prediction to the 3rd level of the EC classification, which may not provide sufficient detail for predicting pesticide degradations. For example the accuracy of EnzymeNet, a residual neural network model, across all the sub-subclasses is 0.398. In addition to that there is no score for the 4th level. Therefore, there is a need for more advanced Deep Learning models that can predict enzyme classes with higher accuracy and resolution, enabling more precise predictions of pesticide degradation pathways. \autocite{watanabeEnzymeNetResidualNeural2023}

Current state-of-the-art models, such as DEEPre or EnzymeNet, have shown promising results in predicting enzyme classes based on sequence data. However, these models are limited in their ability to predict enzyme classes with high specificity and resolution, particularly at the 4th level of the EC hierarchy. The complexity and diversity of enzyme functions at this level pose challenges for accurate prediction, necessitating more advanced Deep Learning models. By leveraging the latest Deep Learning techniques and incorporating detailed biochemical features, it is possible to develop more accurate and efficient models for predicting enzyme classes involved in pesticide degradation.

By contrast, the proposed approach leverages the Deep Learning tool P2RANK to analyze the interactive parts of enzymes, focusing on the ligand-binding sites and the specific amino acids involved. This method can potentially provide a more detailed and accurate prediction of enzyme classes responsible for pesticide degradation, enhancing our understanding of the biodegradation pathways and mechanisms involved. Furthermore, the emphasis on ligand-binding pockets allows for a more nuanced analysis compared to traditional methods that utilize the entire protein sequence. By concentrating on these critical interaction sites, which are crucial for protein functions, P2Rank can identify the specific residues that are directly involved in the catalytic processes. This specificity could not only improves the accuracy of predictions but also reduce the computational complexity by focusing on smaller, more relevant regions of the protein. \autocite{krivakP2RankMachineLearning2018}