%!TEX root = ../Thesis.tex
%% Basierend auf TeXnicCenter-Vorlage von Mark Müller
%%                      Willi Nüßer
%%                      Waldemar Penner     
%%                      Ulrich Reus
%%                      Frank Plass
%%                      Oliver Tribeß 
%%                      Daniel Hintze     
%%%%%%%%%%%%%%%%%%%%%%%%%%%%%%%%%%%%%%%%%%%%%%%%%%%%%%%%%%%%%%%%%%%%%%%

% Wählen Sie die Optionen aus, indem Sie % vor der Option entfernen  
% Dokumentation des KOMA-Script-Packets: scrguide

%%%%%%%%%%%%%%%%%%%%%%%%%%%%%%%%%%%%%%%%%%%%%%%%%%%%%%%%%%%%%%%%%%%%%%%
%% Optionen zum Layout des Artikels                                  %%
%%%%%%%%%%%%%%%%%%%%%%%%%%%%%%%%%%%%%%%%%%%%%%%%%%%%%%%%%%%%%%%%%%%%%%%
\documentclass[%
paper=A4,         % alle weiteren Papierformat einstellbar
fontsize=12pt,    % Schriftgröße (12pt, 11pt (Standard))
BCOR=12mm,         % Bindekorrektur, bspw. 1 cm
DIV=14,            % breiter Satzspiegel
parskip=half*,    % Absatzformatierung s. scrguide 3.1
headsepline,      % Trennline zum Seitenkopf  
%footsepline,     % Trennline zum Seitenfuß
%normalheadings,  % Überschriften etwas kleiner (smallheadings)
listof=totoc,     % Tabellen & Abbildungsverzeichnis ins Inhaltsverzeichnis      
%bibtotoc,        % Literaturverzeichnis im Inhalt 
%draft            % Überlangen Zeilen in Ausgabe gekennzeichnet
footinclude=false,% Fußzeile in die Satzspiegelberechnung einbeziehen 
headinclude=true, % Kopfzeile in die Satzspiegelberechnung einbeziehen 
final             % draft beschleunigt die Kompilierung
]
{scrartcl}

%\setuptoc{toc}{totoc} % Inhaltsverzeichnis ins Inhaltsverzeichnis

% American English
\usepackage[american]{babel}

% Umlaute können verwendet werden
\usepackage[utf8]{inputenc}   

% Echte Umlaute
\usepackage[T1]{fontenc} 

% Latin Modern Font, Type1-Schriftart für nicht-englische Texte
\usepackage{lmodern} 

% 1/2-zeiliger Zeilenabstand
\usepackage[onehalfspacing]{setspace}

% Für die Defenition eigener Kopf- und Fußzeilen
\usepackage{fancyhdr} 

% Für die Verwendung von Grafiken
\usepackage[pdftex]{graphicx}

% Bessere Tabellen
\usepackage{tabularx}

% Für die Befehle \toprule, \midrule und \bottomrule, z.B. in Tabellen 
\usepackage{booktabs}

% Verbessertes URL-Handling mit \url{http://...}
\usepackage{url}

% Listen ohne Abstände \begin{compactlist}...\end{compactlist}
\usepackage{paralist} 

% Ausgabe der aktuellen Uhrzeit für die Draft-Versionen
\usepackage{datetime}

% English quotes
\usepackage{csquotes}

% Konfiguration der Abbildungs- und Tabellenbezeichnungen
\usepackage[format=hang, font={footnotesize, sf}, labelfont=bf, justification=raggedright,singlelinecheck=false]{caption}

% Verbessert die Lesbarkeit durch Mikrotypografie
\usepackage[activate={true,nocompatibility},final,tracking=true,kerning=true,spacing=nonfrench,factor=1100,stretch=10,shrink=10]{microtype}  

% Zitate und Quellenverzeichnis
\usepackage[
    bibstyle=ieee,
    citestyle=ieee,  
    giveninits=false,         % false = Vornamen werden ausgeschrieben
    natbib=true,
    urldate=long,             % "besucht am" - Datum
    %url=false,
    date=long,                
    dashed=false, 
    maxcitenames=3,           % max. Anzahl Autorennamen in Zitaten
    maxbibnames=99,           % max. Anzahl Autorennamen im Quellenverzeichnis
    backend=biber
]{biblatex}

\AtEveryBibitem{%
  \clearfield{issn} % Remove issn
  \clearfield{doi} % Remove doi
  \clearfield{urldate} % Remove urldate
  \ifentrytype{online}{}{% Remove url except for @online
    \clearfield{url}
  }
}
% Bibliograpthy
\bibliography{library/library}

% Keine Einrückung bei einem neuen Absatz 
\parindent 0pt 

% Ebenentiefe der Nummerierung
\setcounter{secnumdepth}{3}

% Gliederungstiefe im Inhaltsverzeichnis 
\setcounter{tocdepth}{3} 

% Tabellen- und Abbildungsverzeichnis mit Bezeichnung:
\usepackage[titles]{tocloft}

% Sourcecode-Listings
\usepackage{listings}

% Bestimmte Warnungen unterdrücken
% siehe http://tex.stackexchange.com/questions/51867/koma-warning-about-toc
\usepackage{scrhack} 

%% http://tex.stackexchange.com/questions/126839/how-to-add-a-colon-after-listing-label
\makeatletter
\begingroup\let\newcounter\@gobble\let\setcounter\@gobbletwo
  \globaldefs\@ne \let\c@loldepth\@ne
  \newlistof{listings}{lol}{\lstlistlistingname}
\endgroup
\let\l@lstlisting\l@listings
\makeatother

\renewcommand*\cftfigpresnum{Figure~}
\renewcommand*\cfttabpresnum{Table~}
\renewcommand*\cftlistingspresnum{Listing~}
\renewcommand{\cftfigaftersnum}{:}
\renewcommand{\cfttabaftersnum}{:}
\renewcommand{\cftlistingsaftersnum}{:}
\settowidth{\cftfignumwidth}{\cftfigpresnum 99~\cftfigaftersnum}
\settowidth{\cfttabnumwidth}{\cfttabpresnum 99~\cftfigaftersnum}
\settowidth{\cftlistingsnumwidth}{\cftlistingspresnum 99~\cftfigaftersnum}
\setlength{\cfttabindent}{1.5em}
\setlength{\cftfigindent}{1.5em}
\setlength{\cftlistingsindent}{1.5em}

\renewcommand\lstlistlistingname{Listing Directory}
 
% Style für Kopf- und Fußzeilenfelder
\pagestyle{fancy}
\fancyhf{}
\fancyhead[R]{\leftmark}
\fancyfoot[R]{\thepage} 
\renewcommand{\sectionmark}[1]{\markboth{#1}{#1}} 
\fancypagestyle{plain}{}

% Macro für Quellenangaben unter Abbildungen und Tabellen
\newcommand{\source}[1]{{\vspace{-1mm}\\\footnotesize\textsf{\textbf{Source:}} \textsf{#1}\par}}

\usepackage{color}
\definecolor{darkred}{rgb}{0.6,0.0,0.0}
\definecolor{darkgreen}{rgb}{0,0.50,0}
\definecolor{lightblue}{rgb}{0.0,0.42,0.91}
\definecolor{orange}{rgb}{0.99,0.48,0.13}
\definecolor{grass}{rgb}{0.18,0.80,0.18}
\definecolor{pink}{rgb}{0.97,0.15,0.45}

% General Setting of listings
\lstset{
  basicstyle=\ttfamily\footnotesize,
  breakatwhitespace=false, 
  breaklines=true,                                   
  keepspaces=true,                                                      
  showspaces=false,                
  showstringspaces=false,
  showtabs=false,                  
  tabsize=2,
  aboveskip=1em,
  abovecaptionskip=-6pt,
  captionpos=b,
  escapeinside={\%*}{*)},
  frame=tb,
  numbers=left,
  numbersep=15pt,
  numberstyle=\scriptsize,
  xleftmargin=2em
}

%\makeatletter
%\def\l@lstlisting#1#2{\@dottedtocline{1}{0em}{1.5em}{\lstlistingname\space{#1}}{#2}}
%\makeatother

% Appendix
\usepackage[nohints]{minitoc} %Appendix

\makeatletter
\newcounter{fktnr}\setcounter{fktnr}{0}
\newcounter{subfktnr}[fktnr]\setcounter{subfktnr}{0}

\renewcommand\thesubfktnr{\arabic{fktnr}.\arabic{subfktnr}}
\newcounter{appendixcounter}
\newcommand{\blatt}{\stepcounter{appendixcounter}}

\newcommand{\Appendix}[1]{\setcounter{appendixcounter}{0}\refstepcounter{fktnr}
\addcontentsline{fk}{subsection}{Appendix~\thefktnr: \hspace*{1em}#1}
\subsection*{{Appendix~\thefktnr \hspace*{1em} #1 \hspace*{-1em}}}
}
\newcommand{\subappendix}[1]{\setcounter{appendixcounter}{0}\refstepcounter{subfktnr}
\addcontentsline{fk}{subsubsection}{Appendix~\thesubfktnr: \hspace*{1em}#1}
\subsubsection*{{Appendix~\thesubfktnr \hspace*{1em} #1 \hspace*{-1em}}}
}

\newcommand{\appendixlist}{\mtcaddsection{\subsection*{Appendix \@mkboth{FKT}{FKT}}}\@starttoc{fk}\newpage}

% Links im PDF
\usepackage[pdfpagemode={UseOutlines}, plainpages=false,breaklinks=true,pdfpagelabels]{hyperref}

% Geometry package
\usepackage{geometry}

% Appendix
\usepackage[toc,page]{appendix}

% Math
\usepackage{amsmath}

% Verbessert das Referenzieren von Kapiteln, Abbildungen etc.
\usepackage{cleveref}

\usepackage{subcaption}

\usepackage{enumitem}

 % Abkürzungsverzeichnis
 \usepackage[automake,
 acronym,         % create list of acronyms
       nonumberlist,
       toc, 
       section,
       nopostdot,  % avoid dot after acronym
       hyperfirst=false,% don't hyperlink first use
       %sanitize=none    % switch off sanitization as description % Deprecated
       ]{glossaries}
       \newglossarystyle{mylist}{%
\setglossarystyle{long}% base this style on the list style
\renewcommand*{\glossaryentryfield}[5]{%
\glsentryitem{##1}\textbf{##2} & ##3 \\}%
}

\makeglossaries\makeglossaries