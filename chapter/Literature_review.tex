\section{Literature Review}
% \addcontentsline{toc}{section}{Literature Review}
\fancyhead[R]{Literature Review}

\subsection{Enzymatic Mechanisms Involved in Pesticide Breakdown}
\label{sec:Enzymatic Mechanisms Involved in Pesticide Breakdown}

The enzymatic degradation of pesticides is essential for understanding their environmental impact and developing bioremediation strategies. Enzymes such as phosphotriesterase (PTE) play a critical role in breaking down organophosphorus pesticides. For instance, PTE can hydrolyze compounds like diazinon and diazoxon. The efficiency of this hydrolysis depends significantly on the structural features of the substrates. Research has demonstrated that PTE facilitates the degradation process through nucleophilic attacks on the phosphorus centers of these pesticides, mediated by specific residues in the enzyme's active site. These findings help in understanding how enzymes can be engineered or utilized for more effective bioremediation applications. \footcite{fuDegradationPesticidesDiazinon2021}

Another significant enzymatic mechanism involves the action of esterases and amidases, which hydrolyze pesticides by breaking down ester and amide bonds. These enzymes, found in both plants and microorganisms, can either detoxify pesticides or, in some cases, activate them, influencing their degradation rates and environmental persistence. For example, specific microbial esterases can hydrolyze various organophosphate insecticides, significantly accelerating their breakdown compared to chemical hydrolysis. \footcite{munneckeEnzymaticHydrolysisOrganophosphate1976}

In addition to these hydrolytic enzymes, oxidative enzymes also play a crucial role in pesticide degradation. Enzymes like cytochrome P450 monooxygenases can oxidize pesticides, making them more water-soluble and thus easier to eliminate from the environment. These enzymes facilitate various oxidation reactions that transform the pesticides into less toxic or more easily degradable forms. For instance, the oxidation of organophosphorus pesticides by cytochrome P450 enzymes has been extensively studied, revealing the molecular mechanisms involved in the biotransformation of these compounds. \footcite{belloTheoreticalApproachMechanism2000}

These enzymatic pathways highlight the potential of leveraging specific enzymes for biotechnological applications, such as bioremediation. By understanding and harnessing these mechanisms, we can develop more efficient methods for detoxifying pesticide-contaminated environments, thus reducing their ecological and health impacts. This is especially relevant at Bayer Crop Science in the context of sustainable agriculture and environmental protection, where the development of enzyme-based biodegradation strategies can contribute to mitigating pesticide residues in soil and water systems.

\subsection{Deep Learning Techniques in Environmental Science}
\label{sec:Deep Learning Techniques in Environmental Science}

Deep learning has emerged as a powerful tool in environmental science, offering advanced methods for predicting and understanding complex biochemical processes. In the context of pesticide degradation, deep learning models can analyze vast amounts of biochemical data to predict enzyme interactions and degradation pathways. There are several deep learning architectures that have been successfully applied to enzyme classification and prediction tasks, providing valuable insights into the mechanisms of pesticide breakdown.

For instance, the DEEPre model uses deep learning to predict enzyme commission (EC) numbers from raw sequence data. This model has shown significant improvements in prediction accuracy over traditional methods by utilizing convolutional and sequential feature extraction techniques. Such models can be crucial for predicting the biodegradation pathways of pesticides, enabling more accurate and efficient environmental risk assessments. \footcite{liDEEPreSequencebasedEnzyme2017}

Another example is the DeEPn model, which uses a deep neural network to classify enzymes into their functional classes, including all seven EC classes. This model has demonstrated high precision and accuracy, making it a valuable tool for environmental scientists looking to understand and predict enzyme-mediated degradation processes. By accurately classifying enzymes, DeEPn facilitates the identification of potential candidates for bioremediation and other environmental applications. \footcite{DeEPnDeepNeural}

Despite the advances made by these models, there is still a need for new approaches to further improve the accuracy and applicability of pesticide degradation predictions. Traditional models often rely on pre-defined features and limited datasets, which can restrict their performance and generalizability. By contrast, my proposed approach leverages the deep learning tool p2rank to analyze the interactive parts of enzymes, focusing on the ligand-binding sites and the specific amino acids involved. \footcite{krivakP2RankMachineLearning2018} This method can potentially provide a more detailed and accurate prediction of enzyme classes responsible for pesticide degradation, enhancing our understanding of the biodegradation pathways and mechanisms involved.